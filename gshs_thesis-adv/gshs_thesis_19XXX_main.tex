% !TeX program = xelatex
%% 부득이하게 pdflatex을 사용해야 할 경우 위의 magic comment를 제거하십시오.

%%%%%%%%%%%%%%%%%%%%%%%%%%%%%%%%%%%%%%%%%%%%%%%%%%%%%%%%%%%%%%%%%%%%%%%%%%%%%%%%%
%%%  LaTeX document class of the thesis of the Gyeonggi Science High School   %%%
%%%  Last edition 2015.11.13 by Chinook Mok                                   %%%
%%%  Continously being modified by gshslatexintro after 2016.02.02.           %%%
%%%  Check the latest version at : latex.gs.hs.kr                             %%%
%%%  Also refer to https://www.facebook.com/gshstexsociety                    %%%
%%%%%%%%%%%%%%%%%%%%%%%%%%%%%%%%%%%%%%%%%%%%%%%%%%%%%%%%%%%%%%%%%%%%%%%%%%%%%%%%%

\documentclass{gshs_thesis}
\RequirePackage{xetexko}
\setmainfont{Times New Roman}
\setmainhangulfont[BoldFont=*,BoldFeatures=FakeBold]{Batang}
\disablecjksymbolspacing
\nonfrenchspacing
\graphicspath{{images/}}
% 이곳에 필요한 별도의 패키지들을 적어넣으시오.
%\usepackage{...}
\usepackage{verbatim} % for commment, verbatim environment
\usepackage{spverbatim} % automatic linebreak verbatim environment
\usepackage{listings}
\lstset{
	basicstyle=\small\ttfamily,
	columns=flexible,
	breaklines=true
}

% -----------------------------------------------------------------------
%                   이 부분은 수정하지 마시오.
% -----------------------------------------------------------------------
\titleheader{졸업논문청구논문}
\school{과학영재학교 경기과학고등학교}
\approval{위 논문은 과학영재학교 경기과학고등학교 졸업논문으로\\
졸업논문심사위원회에서 심사 통과하였음.}
\chairperson{심사위원장}
\examiner{심사위원}
\apprvsign{(인)}
\korabstract{초 록}
\koracknowledgement{감사의 글}
\korresearches{연 구 활 동}

%: ----------------------------------------------------------------------
%:                  논문 제목과 저자 이름을 입력하시오
% ----------------------------------------------------------------------
\title{한글 제목} %한글 제목
\engtitle{English Title} %영문 제목
\korname{홍 길 동} %저자 이름을 한글로 입력하시오 (글자 사이 띄어쓰기)
\engname{Hong, Gil-Dong} %저자 이름을 영어로 입력하시오 (family name, personal name)
\chnname{洪 吉 東} %저자 이름을 한자로 입력하시오 (글자 사이 띄어쓰기)
\studid{14201} %학번을 입력하시오

%------------------------------------------------------------------------
%                  심사위원과 논문 승인 날짜를 입력하시오
%------------------------------------------------------------------------
\advisor{Mok, Chinook}  %지도교사 영문 이름 (family name, personal name)
\judgeone{박 승 원} %심사위원장
\judgetwo{김 대 감}   %심사위원1
\judgethree{목 진 욱} %심사위원2(지도교사)
\degreeyear{2017}   %졸업 년도
\degreedate{2016}{11}{13} %논문 승인 날짜 양식

%------------------------------------------------------------------------
%                  논문제출 전 체크리스트를 확인하시오
%------------------------------------------------------------------------
\checklisttitle{[논문제출 전 체크리스트]} %수정하지 마시오
\checklistI{1. 이 논문은 내가 직접 연구하고 작성한 것이다.} %수정하지 마시오
% 이 항목이 사실이라면 다음 줄 앞에 "%"기호 삽입, 다다음 줄 앞의 "%"기호 제거하시오
\checklistmarkI{$\square$}
%\checklistmarkI{$\text{\rlap{$\checkmark$}}\square$}
\checklistII{2. 인용한 모든 자료(책, 논문, 인터넷자료 등)의 인용표시를 바르게 하였다.} %수정하지 마시오
% 이 항목이 사실이라면 다음 줄 앞에 "%"기호 삽입, 다다음 줄 앞의 "%"기호 제거하시오
\checklistmarkII{$\square$}
%\checklistmarkII{$\text{\rlap{$\checkmark$}}\square$}
\checklistIII{3. 인용한 자료의 표현이나 내용을 왜곡하지 않았다.} %수정하지마시오
% 이 항목이 사실이라면 다음 줄 앞에 "%"기호 삽입, 다다음 줄 앞의 "%"기호 제거하시오
\checklistmarkIII{$\square$}
%\checklistmarkIII{$\text{\rlap{$\checkmark$}}\square$}
\checklistIV{4. 정확한 출처제시 없이 다른 사람의 글이나 아이디어를 가져오지 않았다.} %수정하지 마시오
% 이 항목이 사실이라면 다음 줄 앞에 "%"기호 삽입, 다다음 줄 앞의 "%"기호 제거하시오
\checklistmarkIV{$\square$}
%\checklistmarkIV{$\text{\rlap{$\checkmark$}}\square$}
\checklistV{5. 논문 작성 중 도표나 데이터를 조작(위조 혹은 변조)하지 않았다.} %수정하지 마시오
% 이 항목이 사실이라면 다음 줄 앞에 "%"기호 삽입, 다다음 줄 앞의 "%"기호 제거하시오
\checklistmarkV{$\square$}
%\checklistmarkV{$\text{\rlap{$\checkmark$}}\square$}
\checklistVI{6. 다른 친구와 같은 내용의 논문을 제출하지 않았다.} %수정하지 마시오
% 이 항목이 사실이라면 다음 줄 앞에 "%"기호 삽입, 다다음 줄 앞의 "%"기호 제거하시오
\checklistmarkVI{$\square$}
%\checklistmarkVI{$\text{\rlap{$\checkmark$}}\square$} % usepackage 등의 명령어는 여기에.


\usepackage{tocloft}
\setlength{\cftbeforesecskip}{0pt}
\setlength{\cftbeforesubsecskip}{0pt}
\setlength{\cftbeforesubsubsecskip}{0pt}

\begin{document}
%	\renewcommand\baselinestretch{1.2} % line spacing in the paragraph
	\baselineskip=2.2em         % line spacing in the paragraph
	
	\maketitle  % command to print the title page with above variables
\setcounter{page}{1}
%---------------------------------------------------------------------
%                  영문 초록을 입력하시오
%---------------------------------------------------------------------
\begin{abstracts}     %this creates the heading for the abstract page
\noindent{
Put your abstract here. It is completely consistent with 한글초록.
}
\end{abstracts}

%---------------------------------------------------------------------
%                  국문 초록을 입력하시오
%---------------------------------------------------------------------
\begin{abstractskor}        %this creates the heading for the abstract page
\noindent{
초록(요약문)은 가장 마지막에 작성한다. 연구한 내용, 즉 본론부터 요약한다. 서론 요약은 하지 않는다. 대개 첫 문장은 연구 주제 (+방법을 핵심적으로 나타낼 수 있는 문구: 실험적으로, 이론적으로, 시뮬레이션을 통해)를 쓴다. 다음으로 연구 방법을 요약한다. 선행 연구들과 구별되는 특징을 중심으로 쓴다. 뚜렷한 특징이 없다면 연구방법은 안써도 상관없다. 다음으로 연구 결과를 쓴다. 연구 결과는 추론을 담지 않고, 객관적으로 서술한다. 마지막으로 결론을 쓴다. 이 연구를 통해 주장하고자 하는 바를 간략히 쓴다. 요약문 전체에서 연구 결과와 결론이 차지하는 비율이 절반이 넘도록 한다. 읽는 이가 요약문으로부터 얻으려는 정보는 연구 결과와 결론이기 때문이다. 연구 결과만 레포트하는 논문인 경우, 결론을 쓰지 않는 경우도 있다.
}
\end{abstractskor}


%----------------------------------------------
%   Table of Contents (자동 작성됨)
%----------------------------------------------
\setcounter{secnumdepth}{3} % organisational level that receives a numbers
\setcounter{tocdepth}{3}    % print table of contents for level 3
\tableofcontents


%----------------------------------------------
%     List of Figures/Tables (자동 작성됨)
%----------------------------------------------
\cleardoublepage
\clearpage
\listoffigures	% 그림 목록과 캡션을 출력한다. 만약 논문에 그림이 없다면 이 줄의 맨 앞에 %기호를 넣어서 코멘트 처리한다.

\cleardoublepage
\clearpage
\listoftables  % 표 목록과 캡션을 출력한다. 만약 논문에 표가 없다면 이 줄의 맨 앞에 %기호를 넣어서 코멘트 처리한다.

\cleardoublepage
\clearpage
\renewcommand{\thepage}{\arabic{page}}
\setcounter{page}{1} % Abstract

	%%%%%%%%%%%%%%%%%%%%%%%%%%%%%%%%%%%%%%%%%%%%%%%%%%%%%%%%%%%
	%%%% Main Document %%%%%%%%%%%%%%%%%%%%%%%%%%%%%%%%%%%%%%%%
	%%%%%%%%%%%%%%%%%%%%%%%%%%%%%%%%%%%%%%%%%%%%%%%%%%%%%%%%%%%

	\section{서론}
%\section{Introduction}

\subsection{BibTeX}

BibTeX을 사용하면 참고문헌을 쉽게 정리하거나 관리할 수 있다. 다음은 Google 학술검색 사이트에서 찾은 논문에서 BibTeX 코드를 찾는 방법이다.

\begin{enumerate}
	\item Google 학술검색에서 원하는 논문을 찾는다.
	\item 인용 버튼을 누르면 MLA, APA 등 다양한 인용 스타일이 나온다. 여기서 창 하단의 BibTeX을 클릭한다.
	\item BibTeX을 클릭해서 나온 코드 전문을 복사하여 보고서에 있는 폴더의 bibfile.bib 파일에 붙여넣기 한다.
\end{enumerate}

다음은 예시로 찾은 논문의 BibTeX 코드이다.
\begin{lstlisting}
@article{ward2001landslide,
title={Landslide tsunami},
author={Ward, Steven N},
journal={Journal of Geophysical Research: Solid Earth},
volume={106},
number={B6},
pages={11201--11215},
year={2001},
publisher={Wiley Online Library}
}
\end{lstlisting}

BibTeX에 넣은 논문을 인용하기 위해서는 cite 명령어를 사용한다. 위 코드에서 @article 오른쪽에 있는 문구가 ward2001landslide이므로 이 논문을 인용하기 위해서 명령어
\begin{lstlisting}
\cite{ward2001landslide}
\end{lstlisting}
를 사용한다. 이 명령어 실행 시 명령어를 작성한 자리에 \cite{ward2001landslide}과 같이 논문을 인용했다는 표시가 뜬다. 인용한 논문의 목록은 문서의 References에 표시된다. % Introduction
	
	\include{sub/body1}
	\include{sub/test}
	% Next Section (e.g. Experiment, Linear theory, etc...) 
	% 이외에도 추가로 section마다 파일을 sub 폴더 안에 넣고 여기에서 
	% include 해주면 됩니다.
	% 예시 : methodology.tex을 sub 폴더안에 저장, 이 자리에 
	% \include{sub/methodology} 와 같이 작성
	%%%% 주의
	%%%% 파일이 나뉠 때마다 자동으로 페이지넘김(\clearpage)가 됩니다. 
	%%%% 따라서 subsection을 나누는 용도로는 사용하지 마십시오.
	%%%% \include{sub/experiment} 와 같이...


	\section{결론}
%\section{Conclusion}

\subsection{제목}

내용 % Conclusion
	
	\section{부록}
%\section{Appendix} % Appendix가 없는 경우 주석처리하십시오
	
	\bibliography{bibfile} % 참고문헌
	% BibTeX 코드 쉽게 얻어오는 방법 %
	% Google Scholar 에서 검색한 결과에서 `인용'을 클릭한다.
	% BibTeX 코드를 얻고자 한다면, 하단의 `BibTeX' 을 클릭.
	% 코드가 나온다. Ctrl+A, Ctrl+C로 복사, bibfile에 붙여넣기.
	
	\begin{summary}
\addcontentsline{toc}{section}{Summary}  %%% TOC에 표시
37기부터는 영어로 졸업논문을 작성한 학생은 반드시 5페이지 내외의 한글 요약문을 작성해야 합니다. 한글로 작성하는 학생은 이 부분을 작성하지 않아도 됩니다. gs19xxx.tex에서 주석 처리하십시오.
\end{summary}
 % Summary
	%(영어로 작성한 학생은 이 부분을 주석 처리하십시오.)
	
	\include{sub/footer} % 감사의 글 & 연구활동
\end{document}

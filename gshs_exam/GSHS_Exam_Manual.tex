\documentclass[]{gshs_exam_S}

\usetikzlibrary{patterns,decorations.pathmorphing,arrows.meta}
\usepackage{bm}

\usepackage{tikz-3dplot}

\setlength{\tabcolsep}{0.5em}

\usepackage{fancyvrb}



\makeatletter
\myyear{YYYY}\let\MyYear\@myyear %학년도
\semester{$n$}\let\Semester\@semester %학기
\exams{지필평가}\let\Exams\@exams %1차,2차 지필평가
\subject{\LaTeX\ 양식 사용설명서}\let\Subject\@subject %과목명
\credits{2} %학점
\pfscore{100} %만점
\examtime{60분} %시험시간
\examiner{목진욱} %출제
\reviewer{안재훈} %검토
\makeatother

%% 한글줄간격 %%
\renewcommand{\baselinestretch}{1.3}

\begin{document}

%\maketitle

%%% page 1 %%%
\begin{multicols*}{2}
\noindent\fbox{\parbox{0.98\columnwidth}{
{\sffamily\bfseries 경기과학고 지필평가 \LaTeX\ 양식}
\begin{center}
{\Large\sffamily\bfseries 사용설명서}
\end{center}
\vspace*{-1em}
\begin{flushright}
{\footnotesize Last Update: \today}
\end{flushright}
\begin{enumerate}[leftmargin=5.5mm,label=※]
\item \TeX이 설치되어 있어야 합니다.\\[-2.2em]
\item 수정할 부분을 알려주시면 반영하겠습니다.
\end{enumerate}\vspace{-0.6em}}}\vspace{1em}

\section{\LaTeX를 쓰는 이유} %%%%%%%%%%%%%%%%%%%%%%%%%

\begin{itemize}[leftmargin=5.5mm]
\item 수식이 맵시있다.
\item 간단한 그림은 \TeX에서 바로 입력할 수 있다.
\item 양식을 미리 설정해 두었다면 내용만 입력하면 된다.
\end{itemize}

\section{과목명, 출제자 입력하기} %%%%%%%%%%%%%%%%%%%%%%%%

\subsection{1인 출제}
다음 두 파일이 있어야 한다.
\begin{center}\ttfamily\small
\begin{tabular}{c|c}
\hline
{\footnotesize 양식 설정 파일}& GSHS\_Exam\_S.cls\\
\hline
{\footnotesize 원안지 작성 파일}& GSHS\_Exam\_S(Kor).tex\\
\hline
\end{tabular}
\end{center}
GSHS\_Exam\_S(Kor).tex 파일을 \TeX\ 편집기로 연다. 문서의 preamble에서 다음 부분을 수정한다.
\begin{Verbatim}[frame=single,fontsize=\footnotesize,numbers=left,xleftmargin=5mm]
\makeatletter
\myyear{2018}\let\MyYear\@myyear %학년도
\semester{1}\let\Semester\@semester %학기
\exams{1차 지필평가}\let\Exams\@exams %1차,2차 지필평가
\subject{물리학세미나I}\let\Subject\@subject %과목명
\credits{2} %학점
\pfscore{100} %만점
\examtime{60분} %시험시간
\examiner{목진욱} %출제
\reviewer{안재훈} %검토
\makeatother
\end{Verbatim}
이름이 두 글자인 경우 띄어쓰기로 `{\ttfamily 김 혁}'이라 입력하는 것보다는 `{\ttfamily 김{\textbackslash}hspace\{1em\}혁}'이라고 입력하는 것이 더 좋다.

\subsection{출제자 복수일 때}
다음 이름이 포함된 cls 파일과 tex 파일이 있어야 한다.
\begin{center}\ttfamily\small
\begin{tabular}{c|c|c}
\hline
{\footnotesize 2인 출제}&{\footnotesize 3인 출제}&{\footnotesize 4인 출제}\\
\hline
GSHS\_Exam\_D&GSHS\_Exam\_T&GSHS\_Exam\_Q\\
\hline
\end{tabular}
\end{center}
1인 출제와 마찬가지로 tex 파일의 preamble에서 원안지 정보 파트를 수정한다.

\subsection{문제가 영문으로 작성될 때}
영문으로 작성된 문장이 한글과 동일한 줄 간격을 갖게 되면 여백이 너무 많아져 가독성이 떨어진다. tex 파일의 preamble에서 다음 부분을 찾는다.
\begin{Verbatim}[frame=single,fontsize=\footnotesize,numbers=left,xleftmargin=5mm]
%% 한글줄간격 %%
\renewcommand{\baselinestretch}{1.3}
\end{Verbatim}
2번 줄의 마지막에 있는 숫자를 다소 줄여본다.

\section{문항 작성}

원안지는 2단 구성이므로 각 쪽은 다음과 같은 구조를 갖는다.
\begin{Verbatim}[frame=single,fontsize=\footnotesize,numbers=left,xleftmargin=5mm]
\begin{multicols*}{2}
(1쪽 왼쪽 단)
\vspace*{\fill}
\columnbreak
(1쪽 오른쪽 단)
\end{multicols*}
\begin{multicols*}{2}
(2쪽 왼쪽 단)
\vspace*{\fill}
\columnbreak
(2쪽 오른쪽 단)
\end{multicols*}
\end{Verbatim}
{\ttfamily multicols*} 환경 안에서 문항을 작성하면 된다. 문항은 다음과 같이 {\ttfamily questions} 환경 안에서 작성한다.
\begin{Verbatim}[frame=single,fontsize=\footnotesize,numbers=left,xleftmargin=5mm]
\begin{questions}\extrawidth{8.1em}\setcounter{question}{n}
이 사이에 문항이 작성되어야 함.
문항 번호가 n+1 부터 시작함.
\end{questions}
\end{Verbatim}
{\ttfamily{\textbackslash}extrawidth\{8.1em\}}을 넣어주지 않으면 점수가 문단의 오른쪽 경계 바깥에 나타난다. 이 설정을 해야 점수가 문단 오른쪽 경계의 바로 안쪽에 나타난다. 문제는 이 설정이 페이지 2단 구성을 만나면 다음 페이지에서 문제를 일으킨다는 것이다. 덕분에 각 페이지마다 {\ttfamily questions} 환경을 반복하여 작성해야 한다. {\ttfamily questions} 환경을 종료했다가 다음 페이지에서 다시 시작하면 문항 번호가 초기화 되어 버린다. 따라서 {\ttfamily{\textbackslash}setcounter\{question\}\{n\}}이라는 명령을 추가해야 한다. 여기서 {\ttfamily n}은 앞 페이지의 마지막 문항 번호이다.

\subsection{단답형, 서술형, 논술형}
문항마다 굵은 글씨로 `\textbf{[단답형]}'과 같이 답안 유형을 명시해야 한다. \TeX에서는 단축키가 아닌 명령어로 굵은 글씨를 표현해야 하므로 다소 귀찮음이 있다. 그래서 답안 유형을 다음과 같이 정의하였다.
\begin{center}\ttfamily\small
\begin{tabular}{c|c|c}
\hline
{\footnotesize\textbf{[단답형]}}&{\footnotesize\textbf{[서술형]}}&{\footnotesize\textbf{[논술형]}}\\
\hline
{\textbackslash}ddh&{\textbackslash}ssh&{\textbackslash}nsh\\
\hline
\end{tabular}
\end{center}



\end{multicols*}


%%% page 2 %%%
\begin{multicols*}{2}

\subsection{단독 문항}
문항은 {\ttfamily{\textbackslash}question}이란 명령어로 제작한다. 다음은 소문항 없는 문항 작성 예시이다.
\begin{Verbatim}[frame=single,fontsize=\footnotesize,numbers=left,xleftmargin=5mm]
\question[10] \ssh\ 그림과 같이 질량이 각각 $2\,\mathrm{kg}$, 
$4\,\mathrm{kg}$인 두 물체가 도르래를 걸쳐 실로 연결되어 있다.
도르래의 축에는 $F=60\,\mathrm{N}$의 힘이 연직 상방으로 작용한다. 
도르래의 가속도를 구하시오. (단, 도르래와 실의 질량은 무시하고
중력가속도는 $g=10\,\mathrm{m/s^2}$이다.)
\end{Verbatim}
{\ttfamily{\textbackslash}question} 다음에 등장하는 {\ttfamily [10]}이라는 옵션은 배점을 의미하고, {\ttfamily{\textbackslash}ssh{\textbackslash}}는 미리 약속된 \textbf{[서술형]} 표시이다. {\ttfamily{\textbackslash}ssh} 다음에 {\ttfamily\textbackslash}를 하나 더 넣어주어야 \textbf{[서술형]}과 문제 본문 사이에 띄어쓰기가 반영된다. Xe\LaTeX로 조판하면 예시1과 같은 결과를 얻는다.
\par\vspace{1em}\noindent\tikz{\fill[black](0,0)rectangle(4,-0.1);\draw(0,0)--(10.5,0) node[pos=0.1,right,fill=white]{예시1};}\vspace{-1em}
\begin{questions}\extrawidth{8.1em}
\question[10] \ssh\ 그림과 같이 질량이 각각 $2\,\mathrm{kg}$, $4\,\mathrm{kg}$인 두 물체가 도르래를 걸쳐 실로 연결되어 있다. 도르래의 축에는 $F=60\,\mathrm{N}$의 힘이 연직 상방으로 작용한다. 도르래의 가속도를 구하시오. (단, 도르래와 실의 질량은 무시하고 중력가속도는 $g=10\,\mathrm{m/s^2}$이다.)
\end{questions}
\vspace{-1.4em}\noindent\tikz{\draw(0,0)--(10.5,0);\fill[black] (9,0) rectangle (10.5,0.1);}
배점을 표시하는 방법은 두 가지 방법이 있다. 문제 마지막에 {\ttfamily[10점]}이라고 입력하는 방법과 {\ttfamily{\textbackslash}droppoints}라는 명령어를 넣는 방법이다.
\par\vspace{1em}\noindent\tikz{\fill[black](0,0)rectangle(4,-0.1);\draw(0,0)--(10.5,0) node[pos=0.1,right,fill=white]{예시2: [10점] 직접 입력};}\vspace{-1em}
\begin{questions}\extrawidth{8.1em}\setcounter{question}{0}
\question[10] \ssh\ 그림과 같이 질량이 각각 $2\,\mathrm{kg}$, $4\,\mathrm{kg}$인 두 물체가 도르래를 걸쳐 실로 연결되어 있다. 도르래의 축에는 $F=60\,\mathrm{N}$의 힘이 연직 상방으로 작용한다. 도르래의 가속도를 구하시오. (단, 도르래와 실의 질량은 무시하고 중력가속도는 $g=10\,\mathrm{m/s^2}$이다.) [10점]
\end{questions}
\vspace{-1.4em}\noindent\tikz{\draw(0,0)--(10.5,0);\fill[black] (9,0) rectangle (10.5,0.1);}
\par\vspace{1em}\noindent\tikz{\fill[black](0,0)rectangle(4,-0.1);\draw(0,0)--(10.5,0) node[pos=0.1,right,fill=white]{예시3: {\ttfamily{\textbackslash}droppoints} 사용};}\vspace{-1em}
\begin{questions}\extrawidth{8.1em}\setcounter{question}{0}
\question[10] \ssh\ 그림과 같이 질량이 각각 $2\,\mathrm{kg}$, $4\,\mathrm{kg}$인 두 물체가 도르래를 걸쳐 실로 연결되어 있다. 도르래의 축에는 $F=60\,\mathrm{N}$의 힘이 연직 상방으로 작용한다. 도르래의 가속도를 구하시오. (단, 도르래와 실의 질량은 무시하고 중력가속도는 $g=10\,\mathrm{m/s^2}$이다.) \droppoints
\end{questions}
\vspace{-1.4em}\noindent\tikz{\draw(0,0)--(10.5,0);\fill[black] (9,0) rectangle (10.5,0.1);}
문장의 마침이 페이지 (정확히는 단)의 오른쪽 끝에 가까워 배점을 입력할 여백이 충분치 않다면 예시4와 같이 {\ttfamily{\textbackslash}par{\textbackslash}droppoints}를 사용한다.
\par\vspace{1em}\noindent\tikz{\fill[black](0,0)rectangle(4,-0.1);\draw(0,0)--(10.5,0) node[pos=0.1,right,fill=white]{예시4};}\vspace{-1em}
\begin{questions}\extrawidth{8.1em}\setcounter{question}{0}
\question[10] \ssh\ 그림과 같이 질량이 각각 $2\,\mathrm{kg}$, $4\,\mathrm{kg}$인 두 물체 A, B가 도르래를 걸쳐 실로 연결되어 있다. 도르래의 축에는 $F=60\,\mathrm{N}$의 힘이 중력가속도의 반대 방향으로 작용한다. 도르래의 가속도를 구하시오. (단, 도르래와 실의 질량은 무시하고 마찰은 없으며 중력가속도는 $g=10\,\mathrm{m/s^2}$이다.) \par\droppoints
\end{questions}
\vspace{-1.4em}\noindent\tikz{\draw(0,0)--(10.5,0);\fill[black] (9,0) rectangle (10.5,0.1);}

\vspace*{\fill}
\columnbreak

\subsection{소문항이 있는 문항}

소문항들은 다음과 같이 {\ttfamily{\textbackslash}begin\{parts\}}와 {\ttfamily{\textbackslash}end\{parts\}} 사이에 작성한다. 소문항의 시작은 {\ttfamily{\textbackslash}part}라는 명령어로 시작한다. {\ttfamily{\textbackslash}part} 옆의 {\ttfamily[10]}은 배점을 의미한다. 앞에서와 마찬가지로 {\ttfamily{\textbackslash}droppoints}라는 명령어로 점수를 표시할 수 있다.
\begin{Verbatim}[frame=single,fontsize=\footnotesize,numbers=left,xleftmargin=5mm]
\question 다음 물음에 답하시오.
\begin{parts}
\part[10] \ssh\ 보어의 수소 원자 모델에서 전자의 궤도 반지름은
$r_n =an^2$ ($n$은 자연수)로 주어진다. $a$를 전자의 질량 $m$과
전하량 $e$, 플랑크 상수 $h$, 진공에서의 유전 상수 $\epsilon_0$를
사용하여 나타내시오.\droppoints
\part[10] \ddh\ 수소의 선 스펙트럼은 다음과 같은 리드베리 
공식으로 표현할 수 있다.
\begin{equation*}
\frac{1}{\lambda}
=R\left(\frac{1}{n_1^2}-\frac{1}{n_2^2}\right)
\end{equation*}
여기서 $n_1$, $n_2$ ($>n_1$)는 자연수, $R$은 리드베리 상수이다. 
$R$을 $m$, $e$, $h$, $\epsilon_0$, 그리고 진공에서의 빛의 속력 
$c$를 사용하여 나타내시오.\droppoints
\end{parts}
\end{Verbatim}
\par\vspace{1em}\noindent\tikz{\fill[black](0,0)rectangle(4,-0.1);\draw(0,0)--(10.5,0) node[pos=0.1,right,fill=white]{예시5};}\vspace{-1em}
\begin{questions}\extrawidth{8.1em}\setcounter{question}{1}
\question 다음 물음에 답하시오.
\begin{parts}
\part[10] \ssh\ 보어의 수소 원자 모델에서 전자의 궤도 반지름은
$r_n =an^2$ ($n$은 자연수)로 주어진다. $a$를 전자의 질량 $m$과
전하량 $e$, 플랑크 상수 $h$, 진공에서의 유전 상수 $\epsilon_0$를
사용하여 나타내시오.\droppoints
\part[10] \ddh\ 수소의 선 스펙트럼은 다음과 같은 리드베리 
공식으로 표현할 수 있다.
\begin{equation*}
\frac{1}{\lambda}
=R\left(\frac{1}{n_1^2}-\frac{1}{n_2^2}\right)
\end{equation*}
여기서 $n_1$, $n_2$ ($>n_1$)는 자연수, $R$은 리드베리 상수이다. 
$R$을 $m$, $e$, $h$, $\epsilon_0$, 그리고 진공에서의 빛의 속력 
$c$를 사용하여 나타내시오.\droppoints
\end{parts}
\end{questions}
\vspace{-1.4em}\noindent\tikz{\draw(0,0)--(10.5,0);\fill[black] (9,0) rectangle (10.5,0.1);}
문항의 총 점수를 표기하려면 {\ttfamily{\textbackslash}question} 명령어 앞에 {\ttfamily{\textbackslash}addpoints}를 작성하고, 어미 문항의 마지막에 {\ttfamily{\textbackslash}droptotalpoints}를 넣어주면 된다.
\begin{Verbatim}[frame=single,fontsize=\footnotesize,numbers=left,xleftmargin=5mm]
\addpoints
\question 다음 물음에 답하시오.\droptotalpoints
\end{Verbatim}
\vspace{1em}\noindent\tikz{\fill[black](0,0)rectangle(4,-0.1);\draw(0,0)--(10.5,0) node[pos=0.1,right,fill=white]{예시6};}\vspace{-1em}
\begin{questions}\extrawidth{8.1em}\setcounter{question}{1}
\addpoints
\question[20] 다음 물음에 답하시오.\droptotalpoints
\end{questions}
\vspace{-1.4em}\noindent\tikz{\draw(0,0)--(10.5,0);\fill[black](9,0)rectangle(10.5,0.1);}

\end{multicols*}




\end{document}